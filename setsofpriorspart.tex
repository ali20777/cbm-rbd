\documentclass[authoryear]{elsarticle}

% ------------ packages -------------

\usepackage[utf8]{inputenc}
\usepackage[OT1]{fontenc}
\usepackage{graphicx}
\usepackage[english]{babel}

\usepackage{amsmath}
\usepackage{amsfonts}
\usepackage{amssymb}
\usepackage{amsthm}
\usepackage{bm}

\usepackage[usenames,dvipsnames]{xcolor}
\usepackage{booktabs}
\usepackage{tikz}

\usepackage{url}
\usepackage[bookmarks]{hyperref}

%\usetikzlibrary{shapes.misc,fit}
\usetikzlibrary{%
   arrows,%
   calc,%
   fit,%
   patterns,%
   plotmarks,%
   shapes.geometric,%
   shapes.misc,%
   shapes.symbols,%
   shapes.arrows,%
   shapes.callouts,%
   shapes.multipart,%
   shapes.gates.logic.US,%
   shapes.gates.logic.IEC,%
   er,%
   automata,%
   backgrounds,%
   chains,%
   topaths,%
   trees,%
   petri,%
   mindmap,%
   matrix,%
   calendar,%
   folding,%
   fadings,%
   through,%
   patterns,%
   positioning,%
   scopes,%
   decorations.fractals,%
   decorations.shapes,%
   decorations.text,%
   decorations.pathmorphing,%
   decorations.pathreplacing,%
   decorations.footprints,%
   decorations.markings,%
   shadows}

%\usepackage{hyperref}
%\usepackage[bookmarks]{hyperref}
%\usepackage[colorlinks=true,citecolor=red,linkcolor=black]{hyperref}

% ------------ custom defs -------------

\newcommand{\reals}{\mathbb{R}}
\newcommand{\posreals}{\reals_{>0}}
\newcommand{\posrealszero}{\reals_{\ge 0}}
\newcommand{\naturals}{\mathbb{N}}

\newcommand{\dd}{\,\mathrm{d}}

\newcommand{\mbf}[1]{\mathbf{#1}}
\newcommand{\bs}[1]{\boldsymbol{#1}}
\renewcommand{\vec}[1]{{\bm#1}}

\newcommand{\uz}{^{(0)}} % upper zero
\newcommand{\un}{^{(n)}} % upper n
\newcommand{\ui}{^{(i)}} % upper i

\newcommand{\ul}[1]{\underline{#1}}
\newcommand{\ol}[1]{\overline{#1}}

\newcommand{\Tsys}{T_\text{sys}}

\newcommand{\Rsys}{R_\text{sys}}
\newcommand{\lRsys}{\ul{R}_\text{sys}}
\newcommand{\uRsys}{\ol{R}_\text{sys}}

\newcommand{\fsys}{f_\text{sys}}
\newcommand{\Fsys}{F_\text{sys}}
\newcommand{\lFsys}{\ul{F}_\text{sys}}
\newcommand{\uFsys}{\ol{F}_\text{sys}}

\newcommand{\lgt}{\ul{g}}
\newcommand{\ugt}{\ol{g}}

\newcommand{\E}{\operatorname{E}}
\newcommand{\V}{\operatorname{Var}}
\newcommand{\wei}{\operatorname{Wei}} % Weibull Distribution
\newcommand{\ig}{\operatorname{IG}}   % Inverse Gamma Distribution

\newcommand{\El}{\ul{\operatorname{E}}}
\newcommand{\Eu}{\ol{\operatorname{E}}}

\def\yz{y\uz}
\def\yn{y\un}
%\def\yi{y\ui}
\newcommand{\yfun}[1]{y^{({#1})}}
\newcommand{\yfunl}[1]{\ul{y}^{({#1})}}
\newcommand{\yfunu}[1]{\ol{y}^{({#1})}}

\def\ykz{y\uz_k}
\def\ykn{y\un_k}

\def\yzl{\ul{y}\uz}
\def\yzu{\ol{y}\uz}
\def\ynl{\ul{y}\un}
\def\ynu{\ol{y}\un}
\def\yil{\ul{y}\ui}
\def\yiu{\ol{y}\ui}

\def\ykzl{\ul{y}\uz_k}
\def\ykzu{\ol{y}\uz_k}
\def\yknl{\ul{y}\un_k}
\def\yknu{\ol{y}\un_k}

\newcommand{\ykzfun}[1]{y\uz_{#1}}
\newcommand{\ykzlfun}[1]{\ul{y}\uz_{#1}}
\newcommand{\ykzufun}[1]{\ol{y}\uz_{#1}}

\def\nz{n\uz}
\def\nn{n\un}
%\def\ni{n\ui}
\newcommand{\nfun}[1]{n^{({#1})}}
\newcommand{\nfunl}[1]{\ul{n}^{({#1})}}
\newcommand{\nfunu}[1]{\ol{n}^{({#1})}}

\def\nkz{n\uz_k}
\def\nkn{n\un_k}
\newcommand{\nkzfun}[1]{n\uz_{#1}}
\newcommand{\nkzlfun}[1]{\ul{n}\uz_{#1}}
\newcommand{\nkzufun}[1]{\ol{n}\uz_{#1}}

\def\nzl{\ul{n}\uz}
\def\nzu{\ol{n}\uz}
\def\nnl{\ul{n}\un}
\def\nnu{\ol{n}\un}
\def\nil{\ul{n}\ui}
\def\niu{\ol{n}\ui}

\def\nkzl{\ul{n}\uz_k}
\def\nkzu{\ol{n}\uz_k}
\def\nknl{\ul{n}\un_k}
\def\nknu{\ol{n}\un_k}


\def\taut{\tau(\vec{t})}
\def\ttau{\tilde{\tau}}
\def\ttaut{\ttau(\vec{t})}

\def\MZ{\mathcal{M}\uz}
\def\MN{\mathcal{M}\un}

\def\MkZ{\mathcal{M}\uz_k}
\def\MkN{\mathcal{M}\un_k}

\def\PkZ{\Pi\uz_k}
\def\PkN{\Pi\un_k}
\newcommand{\PZi}[1]{\Pi\uz_{#1}}

\def\tnow{t_\text{now}}
\def\tpnow{t^+_\text{now}}


% ------------ options -------------

\allowdisplaybreaks

\journal{***}

\begin{document}

% ------------ frontmatter -------------

\begin{frontmatter}
\title{Condition-Based Maintenance for Systems based on Bayesian P-Box Component Models***}

\author[tue]{Gero Walter}
\ead{g.m.walter@tue.nl}
\author[tue]{Simme Douwe Flapper}
\ead{s.d.p.flapper@tue.nl}

\address[tue]{School of Industrial Engineering, Eindhoven University of Technology, Eindhoven, Netherlands}


\begin{abstract}
CBM when no (continuous) degradation signal for system available, but status of components observable (works yes/no)

use method from asce-asme paper:
gives RUL in form of a p-box (set of distributions), base maintenance decision directly on that
(via upper expected cost rate)

Bayesian approach allows
component models to include expert info, test data (if available), and status of components in the running system

main message: can do CBM for systems without degradation signal,
especially useful for redundant systems,
method reflects aleatory and epistemic uncertainty in all information sources
\end{abstract}

\begin{keyword}
condition-based maintenance \sep system reliability \sep remaining useful life \sep survival signature \sep imprecise probability \sep p-box
\end{keyword}
\end{frontmatter}

% ------------ manuscript -------------

\section{Introduction}
\label{intro}

main message:
can do CBM for systems without degradation signal,
especially useful for redundant systems,
method reflects aleatory and epistemic uncertainty in all information sources

use method from asce-asme paper:
gives RUL in form of a p-box (set of distributions), base maintenance decision directly on that
(via upper expected cost rate?)

Furthermore, our method appropriately reflects epistemic uncertainty,
by using sets of conjugate priors \citep{diss} for the component models.
The resulting sets of component posteriors are equivalent to parametric p-boxes,
an imprecise probability model popular in risk and reliability engineering.

***figure of process: set of component priors $\to$ set of component posteriors $\to$ system RUL $\to$ $g(\tau)$ $\to$ $\tau^*$

***single $\tau^*$ vs set of $\tau^*$: decision support

***discuss \cite{2015:sankararaman}, epistemic vs. aleatoric uncertainty: big topic in reliability and risk

***refer to Terje Aven, NASA Langley UQ challenge, illustration that uniform distribution does not express ignorance (Edoardo Patelli)***

***we go beyond what \cite{2015:sankararaman} recommends

The proposed system RUL model adequately reflects uncertainty in the RUL distribution
according to the current system status,
in particular making more cautious predictions when failure behaviour in the monitored system
seems to deviate from what is expected from expert knowledge and previous (test) data.

Expert input in our model is, for each component type,
Weibull shape parameter, range of expected failure time (translated to scale parameter),
and range of expert info weight (how sure about mean failure time guess).
Component test data (can include right-censored observations) is optional.
By using ranges we do a systematic sensitivity analysis.

Our model gives lower and upper bounds for the distribution of time till system failure,
taking current (at time $\tnow$) system condition into account.

An advantage of our approach is that
it indicates, for any current time $\tnow$, at which point in the future it is economical to repair the system,
thus allowing for set-up time for maintenance work.

***all expressed in time, but could also be in usage (number of cycles, etc).


\section*{Acknowledgements}

CAMPI

Frank Coolen for inspiring discussions

\section*{Bibliography}

\bibliographystyle{elsarticle-harv}

\bibliography{refs}

\end{document}
