\documentclass[authoryear]{elsarticle}

% ------------ packages -------------

\usepackage[utf8]{inputenc}
\usepackage[OT1]{fontenc}
\usepackage{graphicx}
\usepackage[english]{babel}

\usepackage{amsmath}
\usepackage{amsfonts}
\usepackage{amssymb}
\usepackage{amsthm}
\usepackage{bm}

\usepackage[usenames,dvipsnames]{xcolor}
\usepackage{booktabs}
\usepackage{tikz}

\usepackage{url}
\usepackage[bookmarks]{hyperref}

%\usetikzlibrary{shapes.misc,fit}
\usetikzlibrary{%
   arrows,%
   calc,%
   fit,%
   patterns,%
   plotmarks,%
   shapes.geometric,%
   shapes.misc,%
   shapes.symbols,%
   shapes.arrows,%
   shapes.callouts,%
   shapes.multipart,%
   shapes.gates.logic.US,%
   shapes.gates.logic.IEC,%
   er,%
   automata,%
   backgrounds,%
   chains,%
   topaths,%
   trees,%
   petri,%
   mindmap,%
   matrix,%
   calendar,%
   folding,%
   fadings,%
   through,%
   patterns,%
   positioning,%
   scopes,%
   decorations.fractals,%
   decorations.shapes,%
   decorations.text,%
   decorations.pathmorphing,%
   decorations.pathreplacing,%
   decorations.footprints,%
   decorations.markings,%
   shadows}

%\usepackage{hyperref}
%\usepackage[bookmarks]{hyperref}
%\usepackage[colorlinks=true,citecolor=red,linkcolor=black]{hyperref}

% ------------ custom defs -------------

\newcommand{\reals}{\mathbb{R}}
\newcommand{\posreals}{\reals_{>0}}
\newcommand{\posrealszero}{\reals_{\ge 0}}
\newcommand{\naturals}{\mathbb{N}}

\newcommand{\dd}{\,\mathrm{d}}

\newcommand{\mbf}[1]{\mathbf{#1}}
\newcommand{\bs}[1]{\boldsymbol{#1}}
\renewcommand{\vec}[1]{{\bm#1}}

\newcommand{\uz}{^{(0)}} % upper zero
\newcommand{\un}{^{(n)}} % upper n
\newcommand{\ui}{^{(i)}} % upper i

\newcommand{\ul}[1]{\underline{#1}}
\newcommand{\ol}[1]{\overline{#1}}

\newcommand{\Rsys}{R_\text{sys}}
\newcommand{\lRsys}{\ul{R}_\text{sys}}
\newcommand{\uRsys}{\ol{R}_\text{sys}}

\newcommand{\Fsys}{F_\text{sys}}
\newcommand{\lFsys}{\ul{F}_\text{sys}}
\newcommand{\uFsys}{\ol{F}_\text{sys}}

\def\Rsys{R_\text{sys}}
\def\Tsys{T_\text{sys}}

\newcommand{\E}{\operatorname{E}}
\newcommand{\V}{\operatorname{Var}}
\newcommand{\wei}{\operatorname{Wei}} % Weibull Distribution
\newcommand{\ig}{\operatorname{IG}}   % Inverse Gamma Distribution

\newcommand{\El}{\ul{\operatorname{E}}}
\newcommand{\Eu}{\ol{\operatorname{E}}}

\def\yz{y\uz}
\def\yn{y\un}
%\def\yi{y\ui}
\newcommand{\yfun}[1]{y^{({#1})}}
\newcommand{\yfunl}[1]{\ul{y}^{({#1})}}
\newcommand{\yfunu}[1]{\ol{y}^{({#1})}}

\def\ykz{y\uz_k}
\def\ykn{y\un_k}

\def\yzl{\ul{y}\uz}
\def\yzu{\ol{y}\uz}
\def\ynl{\ul{y}\un}
\def\ynu{\ol{y}\un}
\def\yil{\ul{y}\ui}
\def\yiu{\ol{y}\ui}

\def\ykzl{\ul{y}\uz_k}
\def\ykzu{\ol{y}\uz_k}
\def\yknl{\ul{y}\un_k}
\def\yknu{\ol{y}\un_k}

\newcommand{\ykzfun}[1]{y\uz_{#1}}
\newcommand{\ykzlfun}[1]{\ul{y}\uz_{#1}}
\newcommand{\ykzufun}[1]{\ol{y}\uz_{#1}}

\def\nz{n\uz}
\def\nn{n\un}
%\def\ni{n\ui}
\newcommand{\nfun}[1]{n^{({#1})}}
\newcommand{\nfunl}[1]{\ul{n}^{({#1})}}
\newcommand{\nfunu}[1]{\ol{n}^{({#1})}}

\def\nkz{n\uz_k}
\def\nkn{n\un_k}
\newcommand{\nkzfun}[1]{n\uz_{#1}}
\newcommand{\nkzlfun}[1]{\ul{n}\uz_{#1}}
\newcommand{\nkzufun}[1]{\ol{n}\uz_{#1}}

\def\nzl{\ul{n}\uz}
\def\nzu{\ol{n}\uz}
\def\nnl{\ul{n}\un}
\def\nnu{\ol{n}\un}
\def\nil{\ul{n}\ui}
\def\niu{\ol{n}\ui}

\def\nkzl{\ul{n}\uz_k}
\def\nkzu{\ol{n}\uz_k}
\def\nknl{\ul{n}\un_k}
\def\nknu{\ol{n}\un_k}


\def\taut{\tau(\vec{t})}
\def\ttau{\tilde{\tau}}
\def\ttaut{\ttau(\vec{t})}

\def\MZ{\mathcal{M}\uz}
\def\MN{\mathcal{M}\un}

\def\MkZ{\mathcal{M}\uz_k}
\def\MkN{\mathcal{M}\un_k}

\def\PkZ{\Pi\uz_k}
\def\PkN{\Pi\un_k}
\newcommand{\PZi}[1]{\Pi\uz_{#1}}

\def\tnow{t_\text{now}}
\def\tpnow{t^+_\text{now}}


% ------------ options -------------

\allowdisplaybreaks

\journal{RESS}

\begin{document}

% ------------ frontmatter -------------

\begin{frontmatter}
\title{Condition-Based Maintenance for Systems based on Bayesian Component Models***}

\author[tue]{Gero Walter}
\ead{g.m.walter@tue.nl}
\author[tue]{Simme Douwe Flapper}
\ead{s.d.p.flapper@tue.nl}

\address[tue]{School of Industrial Engineering, Eindhoven University of Technology, Eindhoven, Netherlands}


\begin{abstract}
CBM when no (continuous) degradation signal for system available, but status of components observable (works yes/no)

use method from asce-asme paper:
gives delay time distribution, base maintenance decision directly on that
(via threshold for reliability?)

Bayesian approach allows
component models to include expert info, test data (if available), and status of components in the running system
(write down explicit formula for test data including censored observations)

main message: can do cbm without degradation signal,
useful for redundant systems,
method reflects aleatory and epistemic uncertainty in all information sources
\end{abstract}

\begin{keyword}
condition-based maintenance \sep system reliability \sep delay time model \sep survival signature \sep imprecise probability \sep p-box
\end{keyword}
\end{frontmatter}


% ------------ manuscript -------------

\section{Introduction}
\label{intro}

CBM usually with directly or indirectly observable continuous measurable condition / degradation signal,
delay time (time till failure) via distance of current signal level to fixed known failure threshold
%but in practice threshold often not known

here we propose a different approach to CBM: if no degradation signal for system available,
use system layout (reliability block diagram) and information on components' status:
need only to observe when a component fails,
and expert information about expected component failure times.

i.e., model not for a degradation signal of the system, but for component lifetimes:
we use the well-known Weibull model for component lifetimes.
To keep the model simple, we assume known shape parameter,
and that components fail independently.
these simplifying assumptions might not hold in practice,
then model can serve as approximation.

The Bayesian approach to the Weibull model allows to integrate, for each component type,
three sources of information: expert knowledge, data from component tests,
and status of components in the monitored system, i.e., current system condition.

Furthermore, our method appropriately reflects epistemic uncertainty,
by using parametric p-boxes, an imprecise probability model, for the component models.

epistemic vs. aleatoric uncertainty: big topic in reliability and risk

***NASA Langley UQ challenge, illustration that uniform distribution does not express ignorance (Edoardo Patelli)***

The proposed system delay time model adequately reflects uncertainty in the delay time distribution
according to the current system status,
in particular making more cautious predictions when failure behaviour in the monitored system
seems to deviate from what is expected from expert knowledge and test data.


\section*{Acknowledgements}

CAMPI

%\bibliographystyle{elsarticle-num}
%\bibliography{refs}

\end{document}
